\clearpage{\pagestyle{empty}\cleardoublepage}
\chapter*{Introduzione} 
\markboth{Introduzione}{Introduzione}
\addcontentsline{toc}{chapter}{Introduzione}
\begin{flushright}\begin{small}\textit{"C'è una forza motrice più forte\\ del vapore, dell'elettricità e dell'energia atomica:\\ la volontà."}\\
- Albert Einstein -\\
\end{small}\end{flushright}

Le immagini di microscopia a fluorescenza sono molto più di sole belle immagini, difatti qualsiasi rivista di biologia cellulare mostra che esse sono ormai divenute dati cruciali, sempre più usati come risultati per analisi quantitative.

Ciò è possibile perché la microscopia a fluorescenza fornisce una misura, risolta sia spazialmente che temporalmente, della concentrazione di varie molecole presenti in cellule, tessuti e persino in interi animali. 
L'ampia varietà di etichette molecolari specifiche, tra cui proteine fluorescenti geneticamente codificate, ed una serie di nuove tecniche e modalità di imaging  hanno trasformato tale tipo di microscopia da semplice test di localizzazione a vero e proprio strumento atto all'analisi funzionale quantitativa. 

Inoltre, dato l'ampio utilizzo della microscopia a fluorescenza ed il forte sviluppo di nuove metodologie, è diventato fondamentale per la maggior parte dei biologi essere in grado di valutare criticamente le immagini acquisite, facendone un'analisi quantitativa, la più corretta possibile \cite{fluo}.

