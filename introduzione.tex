\clearpage{\pagestyle{empty}\cleardoublepage}

\chapter*{Introduzione} 

\markboth{Introduzione}{Introduzione}
\addcontentsline{toc}{chapter}{Introduzione}
\begin{flushright}\begin{small}\textit{"C'è una forza motrice più forte\\ del vapore, dell'elettricità e dell'energia atomica:\\ la volontà."}\\
- Albert Einstein -\\
\end{small}\end{flushright}

Nella moderna biologia cellulare le immagini a microscopia a fluorescenza stanno divenendo via via più importanti, in quanto permettono di ottenere risultati necessari per successive analisi quantitative.
Ciò è possibile poiché tale tipo di microscopia fornisce una misura della concentrazione di varie molecole presenti in cellule e tessuti, risolta sia spazialmente che temporalmente.
L'ampia varietà di etichette molecolari specifiche, tra cui proteine fluorescenti geneticamente codificate, ed una serie di nuove tecniche e modalità di imaging hanno trasformato la microscopia a fluorescenza da semplice test di localizzazione a vero e proprio strumento atto all'analisi funzionale quantitativa.
Inoltre, dato l'ampio utilizzo di tale tecnica ed il rapido sviluppo di nuove metodologie, è diventato fondamentale per la maggior parte dei biologi essere in grado di valutare criticamente le immagini acquisite, effettuando un'analisi quantitativa nel modo più corretto possibile.

Tuttavia, per poter eseguire analisi quantitative su immagini in fluorescenza è necessaria una preliminare calibrazione del microscopio.
Difatti, le immagini a microscopia soffrono di varie distorsioni, dovute a molteplici cause, tra cui aberrazioni, diffrazione e difetti di natura tecnico-sperimentale.

Nel lavoro svolto all'interno di questa tesi ci siamo occupati in particolare della correzione di uno di questi possibili difetti dell'immagine: la disomogeneità di campo, ovvero una non uniforme fluorescenza dovuta alla forma irregolare del fascio di eccitazione. 
Per conseguire l'obiettivo da noi proposto è stato necessario l'utilizzo di materiali nanometrici con fluorescenza nota, in modo da ottenere apposite immagini di riferimento. 
A partire da queste, con delle procedure di \textit{image processing} da noi implementate, abbiamo stimato la funzione di correzione della fluorescenza, localmente per ogni punto dell'immagine.

Nel primo capitolo è introdotto il fenomeno fisico della fluorescenza ed è mostrata la sua applicazione nel campo biologico, grazie ai cosiddetti fluorofori, o fluorocromi, inseriti all'interno delle cellule secondo opportune tecniche di marcatura.

Nel secondo capitolo sono descritte le varie caratteristiche di un tipico microscopio a fluorescenza, ponendo in particolare l'attenzione sul microscopio Nikon Eclipse-Ti del Dipartimento di Fisica di Bologna. 
Inoltre, sempre in tale capitolo, sono introdotte le principali imperfezioni che possono manifestarsi all'interno delle immagini a microscopia, suddivisibili a seconda della causa in difetti di natura geometrica, fisica o tecnica.

Nel terzo capitolo è descritto nei suoi tratti fondamentali l'algoritmo sviluppato con tale progetto di tesi. 
Esso, scritto nel linguaggio di programmazione Python, può essere frazionato in tre principali fasi: correzione della disomogeneità spaziale della fluorescenza, rimozione della luminosità di background e correzione finale dell'immagine volta ad analisi quantitativa.

Infine, nel quarto capitolo sono esposti i risultati ottenuti tramite l'applicazione dell'algoritmo su immagini a fluorescenza rossa di cellule di fibroblasti.
Essi sono stati analizzati e rivisitati sotto differenti punti di vista, da quelli di natura qualitativa fino a quelli più quantitativi come l'analisi del p-value o della matrice di correlazione, così da valutare al meglio l'efficacia del software di correzione da noi sviluppato.
