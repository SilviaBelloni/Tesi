\clearpage{\pagestyle{empty}\cleardoublepage}
\chapter*{Introduzione} 
\markboth{Introduzione}{Introduzione}
\addcontentsline{toc}{chapter}{Introduzione}
\begin{flushright}\begin{small}\textit{"C'� una forza motrice pi� forte\\ del vapore, dell'elettricit� e dell'energia atomica:\\ la volont�."}\\
- Albert Einstein -\\
\end{small}\end{flushright}

Le immagini di microscopia a fluorescenza sono molto pi� di sole belle immagini, difatti qualsiasi rivista di biologia cellulare mostra che esse sono ormai divenute dati cruciali, sempre pi� usati come risultati per analisi quantitative.
Ci� � possibile perch� la microscopia a fluorescenza fornisce una misura, risolta sia spazialmente che temporalmente, della concentrazione di varie molecole presenti in cellule, tessuti e persino in interi animali. 
L'ampia variet� di etichette molecolari specifiche, tra cui proteine fluorescenti geneticamente codificate, ed una serie di nuove tecniche e modalit� di imaging  hanno trasformato tale tipo di microscopia da semplice test di localizzazione a vero e proprio strumento atto all'analisi funzionale quantitativa. Inoltre, dato l'ampio utilizzo della microscopia a fluorescenza ed il forte sviluppo di nuove metodologie, � diventato fondamentale per la maggior parte dei biologi essere in grado di valutare criticamente le immagini acquisite, facendone un'analisi quantitativa, la pi� corretta possibile \cite{fluo}.