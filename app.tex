\clearpage{\pagestyle{empty}\cleardoublepage}
\chapter{Codice Sorgente} 
\label{appendiceWSS} 

\lstset{language=Python, numbers=left, stepnumber=1, breaklines=true}

\begin{lstlisting}
from __future__ import division
import pylab
import numpy as np
import scipy
import scipy.ndimage as ndimage
import scipy.ndimage.filters as filters
import matplotlib.pyplot as plt
import itertools
import scipy.stats as st
import numexpr as ne
import PyZenity
from pylab import *
from scipy.optimize import curve_fit
from mpl_toolkits.mplot3d import Axes3D
from scipy.stats.mstats import mquantiles
from scipy.ndimage.filters import gaussian_filter
from sklearn import mixture


# DEFINIZIONE DELLE FUNZIONI ----------
# Calcolo dell'intensita (integrale media) associata ad un massimo 
def intensity(x, y, imm):
    z = []    
    delta = 5
    for xi, yi in zip(x, y):
        val = 0
        count = 0
        for i in range(-delta, delta):
            for j in range(-delta, delta):
                if imm[yi+i,xi+j]>0:
                    count=count+1
                    val = val + imm[yi+i, xi+j]
        z.append(val/count)  
    z = np.array(z)
    return z
        
# Funzione di ricerca dei massimi 
def find_max(imm):
    vicinanza_size = 10
    imm_max = filters.maximum_filter(imm, vicinanza_size)
    maxima = (imm == imm_max)
    imm_min = filters.minimum_filter(imm, vicinanza_size)
    soglia = 40
    diff = ((imm_max - imm_min) > soglia)
    maxima[diff == 0] = 0
    labeled, num_objects = ndimage.label(maxima)
    slices = ndimage.find_objects(labeled)
    margine = 50
    x, y = [], []
    for dy,dx in slices:
        x_center = (dx.start + dx.stop - 1)//2
        y_center = (dy.start + dy.stop - 1)//2 
        if x_center<margine or x_center>imm.shape[1]-margine:
            continue                                         
        if y_center<margine or y_center>imm.shape[0]-margine:
            continue
        x.append(x_center)
        y.append(y_center)
    x = np.array(x)
    y = np.array(y)
    z = intensity(x, y, imm)
    return x, y, z

# Ricerca della funzione ''gauss'' che fitta al meglio i massimi
def gauss(x, y, dsx, dsy, maxint, corr, cx, cy, bg, e):
        if dsx<0:
            return np.nan
        if dsy<0:
            return np.nan
        if maxint<0:
            return np.nan
        if cx<0:
            return np.nan
        if cy<0:
            return np.nan
        if bg<0: 
            return np.nan
        if ((corr<-1) or (corr>1)):
            return np.nan
        if e<0:
            return np.nan
        varx =((x-cx)/dsx)**2
        vary = ((y-cy)/dsy)**2
        varxy = (corr*(x-cx)*(y-cy))/(dsy*dsx)
        d = sqrt(varx+vary+varxy)
        return ne.evaluate('bg + maxint * exp( -0.5 * d**e )')
        
# Ricerca dei parametri della curva ''gauss''
def analyze(imm):
    x, y, z = find_max(imm)
    xy = array(zip(x, y))
    def gauss_for_curve(xy, *args):
        x, y = xy.T
        return gauss(x, y, *args)
    parametri_0 = (800, 600, np.max(imm), 0, 800, 600, 0, 2)
    parametri, covar = curve_fit(gauss_for_curve, xy, z, parametri_0)
    imm_normalized = correction(imm, parametri)
    # Esclusione dei punti più intensi
    quantile = mquantiles(imm_normalized[y, x], 0.95)
    newx, newy = [], []
    for (valore, xtemp, ytemp) in zip(imm_normalized[y, x], x, y):
        if valore < quantile:
            newy.append(ytemp) 
            newx.append(xtemp)
    newx = asarray(newx) 
    newy = asarray(newy)
    newz = intensity(newx, newy, imm)
    newxy = array(zip(newx, newy))
    newparametri, newcovar = curve_fit(gauss_for_curve, newxy, newz, parametri)
    print newparametri
    return newparametri

# Correzione dell'immagine sulla base dei parametri della curva ''gauss''
def correction(imm, par):

    xx, yy = np.meshgrid(np.linspace(0, 1600, 1600), np.linspace(0, 1200, 1200))
    
    zz = gauss(xx, yy, *par)
    
    zz_normalized = zz / np.max(zz)
    imm_normalized = imm/zz_normalized
    return imm_normalized 
# -------------------------------------


# Fase I: EFFETTO DEI BORDI - IMMAGINE DELLE SFERETTE AD UNA SOLA INTENSITA
# Inserimento immagine da utente
fname_sfere = PyZenity.GetFilename()[0]
imm = pylab.imread(fname_sfere)
# Applicazione filtro gaussiano
imm = gaussian_filter(imm, 2)
# Sottrazione della prima approssimazione del background
mode = max(scipy.stats.mode(imm)[0][0])
imm = imm - mode
# Identificazione dei centri delle sferette
xi, yi, zi = find_max(imm)
# Ricerca dei parametri della curva ''gauss''
par = analyze(imm)
# Autocorrezione dell'immagine delle sferette ad una sola intensita
corretta = correction(imm, par)
x, y, z = find_max(corretta)


# Fase II: LINEARITA - IMMAGINE DELLE SFERETTE CON MIX DI INTENSITA
# Inserimento immagine da utente
fname_sferemix = PyZenity.GetFilename()[0]
imm1 = pylab.imread(fname_sferemix)
# Applicazione filtro gaussiano e sottrazione della prima approssimazione del background
imm1 = gaussian_filter(imm1, 2)
imm1 = imm1 - mode
# Identificazione dei centri delle sferette
x1i, y1i, z1i = find_max(imm1)
# Correzione dell'immagine delle sferette con mix di intensita
corretta1 = correction(imm1, par)
x1, y1, z1 = find_max(corretta1)
# Studio della risposta lineare con il gaussian mixture
clf = mixture.GMM(n_components = 5, n_init = 100, init_params = '', params = 'mwc') 
clf.weights_ = asarray([0.2]*5)
clf.means_ = asarray((3.0, 4.0, 5.0, 6.0, 7.0)).reshape(5, 1)
clf.covars_ = asarray([0.015]*5).reshape(5,1)
clf.fit(log(z1))
intnote = log([1, 3, 10, 33, 100])
parlin = st.linregress(exp(intnote), exp(clf.means_.ravel()))
def linreg(x, A, B):
    return A*x+B
x = exp(intnote)
y = exp(clf.means_.ravel())
erry = exp(clf.covars_.ravel())
param0 = [parlin[0], parlin[1]]
param, varpar = curve_fit(linreg, x, y, p0 = param0, sigma = erry)
errpar = (diag(varpar))**(1/2)
corretta1 = corretta1 - param[1]


# CORREZIONE DELL'IMMAGINE DI CELLULE A FLUORESCENZA
# Inserimento immagine da utente
cell = PyZenity.GetFilename()[0]
imm_cell_0 = pylab.imread(cell)
# Applicazione filtro gaussiano e sottrazione della prima approssimazione del background
imm_cell = gaussian_filter(imm_cell_0, 2)
imm_cell = imm_cell - mode
# Risposta lineare: sottrazione del parametro di background rivelato dalla fase II
imm_cell_corr = imm_cell - param[1]
# Effetto dei bordi: correzione tramite i parametri calcolati nella fase I
imm_cell_corr = correction(imm_cell_corr, par)
# Salvataggio dell'immagine delle cellule corretta
scipy.misc.imsave('Corretta.tif', imm_cell_corr)
\end{lstlisting}