\clearpage{\pagestyle{empty}\cleardoublepage}

\chapter*{Conclusioni}

\begin{flushright}
\begin{small}\textit{"Computers are useless.\\
 They can only give you answers."}\\
- Pablo Picasso -\\
\end{small}\end{flushright}

\markboth{Conclusioni}{Conclusioni}
\addcontentsline{toc}{chapter}{Conclusioni}

In questo lavoro di tesi ci siamo proposti di calibrare il microscopio a fluorescenza usato nel laboratorio di biofisica del Dipartimento di Fisica di Bologna, così da poter effettuare sulle immagini con esso acquisite eventuali analisi quantitative. 
Nello specifico ci siamo proposti di correggere il difetto di non omogeneità d'illuminazione, dovuto alla forma irregolare del fascio di eccitazione. 

Tale correzione è basata sull'acquisizione, in parallelo al campione sotto esame, di immagini di calibrazione generate tramite sfere nanometriche a fluorescenza nota.
Parte del lavoro di tesi è infatti consistita nella generazione di diverse immagini di calibrazione, così da poter verificare a posteriori il nostro metodo di ``image processing''.

La correzione della luminosità è stata effettuata tramite un algoritmo da noi implementato nel linguaggio di programmazione Python.
Per la creazione di tale software di elaborazione d'immagini abbiamo ipotizzato una possibile distribuzione dell'intensità dovuta alla non omogeneità del fascio ed abbiamo quindi stimato i parametri tramite un'apposita procedura di maximum likelihood.
Per effettuare questa stima abbiamo anche tenuto conto di possibili effetti dovuti alla luminosità di background, al margine ed alla sovrapposizione di più probes.
Questa procedura è stata ripetuta su quattro diverse immagini di calibrazione, per valutarne la consistenza e l'efficacia.

Per poter verificare che la procedura ideata abbia le desiderate proprietà di linearità fra segnale misurato ed intensità nota, ci siamo serviti di un'ulteriore immagine di calibrazione contenente una mistura di sfere nanometriche con intensità variabili su due ordini di grandezza.
Grazie ad essa abbiamo verificato che la nostra procedura incrementa la discriminazione tra queste intensità e che il segnale risultante può essere facilmente mappato su una scala lineare. 

Dall'analisi dei risultati ottenuti, la nostra procedura di calibrazione sembra dare una risposta coerente fra diverse immagini di calibrazione ed un segnale compatibile con le proprietà desiderate di linearità e di indipendenza dalla posizione all'interno dell'immagine. 
Sulla base di tali considerazioni è possibile affermare che questo algoritmo verrà inserito in un programma che permetterà la calibrazione delle immagini in microscopia a fluorescenza nei futuri esperimenti del laboratorio di biofisica.

Tuttavia, per ottenere un'effettiva misura quantitativa della fluorescenza osservata sarà in futuro necessario valutare altri parametri di calibrazione.
Per prima cosa sarà necessario ripetere l'acquisizione di immagini di calibrazione su più esperimenti, così da decidere se sia necessaria una calibrazione individuale per ciascun esperimento o se sia possibile una calibrazione generale, indipendentemente dal singolo esperimento.
Sarà anche importante valutare l'allineamento fra le immagini acquisite in campo chiaro e con diversi tipi di fluorescenza e stimare la Point Spread Function (PSF) del microscopio su ciascuna frequenza, per poter effettuare una deconvoluzione dell'immagine in grado di ridurre la perdita di risoluzione dovuta alla diffrazione.
