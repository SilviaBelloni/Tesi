\clearpage{\pagestyle{empty}\cleardoublepage}

\chapter*{Conclusioni}

\begin{flushright}
\begin{small}\textit{"Computers are useless.\\
 They can only give you answers."}\\
- Pablo Picasso -\\
\end{small}\end{flushright}

\markboth{Conclusioni}{Conclusioni}
\addcontentsline{toc}{chapter}{Conclusioni}

Obiettivo di questa tesi è la progettazione ed implementazione di un algoritmo di correzione e calibrazione di immagini generate tramite microscopia a fluorescenza.
L'algoritmo è stato calibrato usando il microscopio a fluorescenza presente nel laboratorio di biofisica del Dipartimento di Fisica ed Astronomia di Bologna.
Questa correzione ha lo scopo di migliorare l'analisi quantitativa di tali immagini, attualmente resa poco attendibile dalle varie imperfezioni ottiche presenti.
Nello specifico ci siamo proposti di correggere il difetto di non omogeneità d'illuminazione, dovuto alla forma irregolare del fascio di eccitazione, e la rimozione della fluorescenze residua di background.

Tale correzione è basata sull'acquisizione, in parallelo al campione sotto esame, di immagini di calibrazione contenenti sfere nanometriche a fluorescenza nota.
Parte del lavoro di tesi è consistita nella generazione di diverse di queste immagini, così da poter implementare e verificare il nostro metodo di \textit{image processing}.

La correzione della luminosità è stata effettuata tramite un algoritmo da noi implementato nel linguaggio di programmazione Python.
Per la creazione di tale software di elaborazione d'immagini abbiamo ipotizzato una possibile distribuzione dell'intensità dovuta alla non omogeneità del fascio ed abbiamo quindi stimato i parametri tramite un'apposita procedura di maximum likelihood.
Per effettuare questa stima abbiamo anche tenuto conto di possibili effetti dovuti alla luminosità di background, alla sovrapposizione di più nanosfere e ad effetti di bordo nel corso dell'elaborazione.
Questa procedura è stata ripetuta su quattro diverse immagini di calibrazione, per valutarne la consistenza e l'efficacia.

Per verificare che la procedura ideata abbia le desiderate proprietà di linearità fra segnale misurato ed intensità nota, ci siamo serviti di un'ulteriore immagine di calibrazione contenente una mistura di sfere nanometriche con intensità variabili su due ordini di grandezza.
Grazie ad essa abbiamo verificato che la nostra procedura incrementa la discriminazione tra queste intensità e che il segnale risultante può essere facilmente mappato su una scala lineare. 

Dall'analisi dei risultati ottenuti, la nostra procedura di calibrazione sembra dare una risposta coerente fra diverse immagini di calibrazione ed un segnale compatibile con le proprietà desiderate di linearità e di indipendenza dalla posizione all'interno dell'immagine. 
Questo algoritmo verrà quindi inserito in un programma che permetterà la calibrazione delle immagini in microscopia a fluorescenza nei futuri esperimenti del laboratorio di biofisica.

Per un futuro completamente della calibrazione del microscopio, successivamente a questo lavoro i tesi, sarà necessario includere addizionali metodi di correzione per tener conto di altre fonti di incertezza.
Per prima cosa bisognerà ripetere l'acquisizione di immagini di calibrazione su più esperimenti, così da decidere se sia necessaria una calibrazione individuale per ciascun esperimento o se sia possibile una calibrazione generale, indipendentemente dal singolo esperimento.
Dovranno essere valutati anche gli effetti di distorsione legati al sistema ottico, come aberrazioni sferiche, coma, astigmatismo e distorsione.
Sarà anche importante valutare l'allineamento fra le immagini acquisite in campo chiaro e con diversi tipi di fluorescenza e stimare la Point Spread Function (PSF) del microscopio su ciascuna frequenza, per poter effettuare una deconvoluzione dell'immagine in grado di ridurre la perdita di risoluzione dovuta alla diffrazione.
